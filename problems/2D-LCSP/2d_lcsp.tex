\documentclass[]{article}
\usepackage{verbatim}
\usepackage{amsmath}
\usepackage{graphicx}
%opening
\title{\textbf{Problem statement: The 2D-LCS Problem} }
\author{Marko Djukanović}

\begin{document}

\maketitle

%\begin{abstract}
%\end{abstract}


\section{The 2D-Longest Common Subsequence Problem}

\subsection{Introduction}

The Longest Common Subsequence (LCS) problem is a fundamental optimization problem arising in bioinformatics and computational biology that rates the structural similarity between two or more biological sequences.
It has applications data compression, text editing, etc.  

The literature is known for many linear generalizations of the LCS problem. The first inherently multidimensional generalization of the LCS problem is provided in~\cite{2d_lcs_amir} by introducing the concept of \textit{multidimensional} \textit{sequences}, so-called 2D-sequences. Comparing such sequences is of practical interest in, for example, quantifying similarities of two images represented by respective matrices of pixels. 

%Provide a brief introduction of the problem, possibly including its motivation and context. This should be a short (2 or 3 paragraphs) high-level description.

\subsection{Task}

In the 2D-LCS problem, given are two two-dimensional matrices or multidimensional objects. The task is 
to  rate the (structural) similarity
between a pair of two-dimensional objects.
To do so, one may seek identical symbols in both matrices, preserving their order. This will not always result in a sub-matrix, but rather the
symbols common to both matrices, which preserve the order relation between them in both matrices.   



%Describe the high-level optimization task in one or two sentences.

\subsection{Problem statement}

Given are two matriice  $A$ and $B$ of dimensions $n\times n$ and $m \times m$ respectively, over a finite alphabet $\Sigma$.  \\
The objective is to find a \textbf{maximum} domain size 1-1 mapping $f \colon [n] \times [n] \mapsto [m] \times [m] $ so that $A[i, j] = B[f(i, j)]$ for each $(i, j) \in Dom(f)$ and
for each two elements $(i, j), (i',j') \in Dom(f)$, the following conditions hold:
\begin{enumerate}  
	\item $i < i'$ iff $f(i) < f(i')$  \label{constr-1}
	\item $j < j'$ iff $f(j) < f(j')$
	\item $i = i'$ iff $f(i) = f(i')$
    \item $j = j'$ iff $f(j) = f(j')$	\label{constr-4}
\end{enumerate}


A feasible \textit{solution} is therefore any (1-1) mapping $f$ from $[n]^2$ to $[m]^2$ satisfying Conditions \ref{constr-1}--\ref{constr-4}. 

\textit{The objective value} of a (feasible) solution $f$:  cardinalty of the domain size of $f$, i.e. the value $|Dom(f)|$. 

It is proven that the 2D-LCS problem is $\mathcal{NP}$-hard, see~\cite{2d_lcs_amir}. 

  

\subsection{Instance data file}


Each \textit{problem instance} is represented by two matrices of characters of dimensions $n \times n$ and $m \times m$, respectively. These two numbers stating the dimensions may be written in the first line of the input file.  The subsequent $n$ lines are in charge for storing  elements of the first matrix row-by-row. The last $m$ lines store the elements of the second matrix row-by-row.

The extension of the file can be .\textit{txt}, .\textit{csv}, or some other similar (textual) format.

\subsection{Solution file}

A solution can be stored in a textual-type file where  in the first line elements of the domain are stored, given as a sequence of pairs $(i, j)$.  The second line keeps respective co-domain values of the solution also as a sequence of pairs $(f(i), f(j))$. 

\subsection{Example}
For the shake of clarity, given is an example in the subsequent section. 

\subsubsection{Instance}

Input file: 
\begin{verbatim}
3 3
A C C
B D A 
A A C
A A C
D B C
B D D
\end{verbatim}
Visually, the instance correspond to the matrices shown in Figure~\ref{fig:instance}. 

\begin{figure}[ht]
	\centering
	\includegraphics[width=150pt,height=60pt]{example\_2d\_lcs.png}
	\caption{The symbols in bold denote a solution to the instance\protect\footnotemark }
	\label{fig:instance}
\end{figure}

\footnotetext{The example borrowed from~[1]}

  

\subsubsection{Solution}

An optimal solution to the problem problem instance given above is a function $f:X \rightarrow Y$ with 
$$X=\{ (1, 1), (1, 2), (2, 1), (2, 2)   \}, Y=\{ (1, 1), (1, 3), (3, 1), (3, 3) \}.$$
So, the objective value of $f$ is equal to 4. 

%Provide a feasible solution to the example instance in the described format (including its evaluation measure).

\subsubsection{Explanation}
 One can easily verify that the solution is feasible (after directly verifying Conditions 1-4). It can be easily verified that one cannot choose all (3) symbols of a row (or a column) to be elements of any feasible function $f$.  Thus, an upper bound of the optimal solution must be less or equal to 4. 

%Optionally, provide a descriptive and/or visual explanation of the solution (and its evaluation measure value) for the instance.

\bibliography{plain}


\begin{thebibliography}{9}
	\bibitem{2d_lcs_amir}
	Amir, A., Hartman, T., Kapah, O., Shalom, B. R., \& Tsur, D. (2007, October). Generalized {LCS}. In International Symposium on String Processing and Information Retrieval (pp. 50-61). Berlin, Heidelberg: Springer Berlin Heidelberg.
	
 
\end{thebibliography}

\end{document}
